\documentclass[a4paper,twoside,12pt]{book}
\usepackage{natbib}
\usepackage[utf8]{inputenc}
\usepackage[T1]{fontenc}
\usepackage{newcent}
\usepackage[spanish]{babel}

\addto{\captionsspanish}{\def\chaptername{}}
% para que no aparezaca la palabra capítulo.
\usepackage{rotating}
\usepackage{xspace}
%\usepackage{xy}
\usepackage{amsmath}
\usepackage{framed}
\usepackage{type1cm}
\usepackage{textcomp} %para \textcelsius
\usepackage[indent,small,it,bf]{caption}
%formato de los captions (pies de tablas y figuras): ``using imported graphics in latex (pág 50).
\usepackage[tight]{minitoc}
\setcounter{minitocdepth}{2}
% tablas de contenidos en cada capitulo.
\bibliographystyle{fany}
\usepackage[nottoc,notlot,notlof]{tocbibind}
% para que la bibliografía salga en el índice y no lo haga el lot y el lof.
\usepackage{multicol}
% para juntar columnas en las tablas.
\usepackage{marvosym,amssymb,url}
% para insertar símbolos como \Male y \Female.
\usepackage{setspace,paralist,mdwlist,graphicx}
%paquetes para insertar imágenes.
\usepackage{afterpage}
%para colocar un float en la pagina siguiente
\usepackage[table]{xcolor}
%permite poner colores en las tablas
\usepackage{multirow}
%para juntar filas en las tablas
\usepackage{rotating}
%para rotar texto

\usepackage{texshade}
\usepackage{longtable}
%\usepackage[paperwidth=17.5cm,paperheight=25cm,top=3cm,bottom=2.75cm,inner=3cm,outer=2.5cm,dvips=false,pdftex=false,vtex=false]{geometry}
\usepackage[]{geometry}
\geometry{
%paperwidth=17.5cm,
%paperheight=25cm,
top=3cm,
bottom=3.25cm,
inner=3.25cm,
outer=2.5cm
}

%\usepackage[frame,a4,center,dvips]{crop} %ejes
\usepackage[a4,center,dvips]{crop} %centrada
%\usepackage{crop} %esquina


\definecolor{blacklinkcolor}{rgb}{0,0,0}
\definecolor{azul}{rgb}{0,0,1}
\definecolor{verde}{rgb}{0,0.6,0}
\definecolor{rojo}{rgb}{1,0,0}
%\usepackage[dvips,ps2pdf,breaklinks,citecolor=azul,linkcolor=verde]{hyperref}
\usepackage[dvips,ps2pdf,breaklinks,citecolor=blacklinkcolor,linkcolor=blacklinkcolor]{hyperref}
\hypersetup{%
colorlinks=true,%
urlcolor=black%
}



%\usepackage{tocloft}
%para personalizar la toc.

\usepackage{epsfig}
%para que latex pueda rotular las figuras.
\usepackage{texshade}
%(para crear los alineamientos de secuencias)
\usepackage{hhline}
%\usepackage{xcolor}
% colorear texto.

\usepackage{fancyvrb}

\date{17 Julio de 2013}
\newcommand{\titulo}{Caracterización de un ratón \emph{CreERT2} para la inactivación de alelos condicionales \emph{flox} en células de Sertoli adultas}

\title{\titulo}

\author{Alicia Hurtado Madrid}
%\setlength{\parskip}{.7ex}
\setlength\parskip{.5\baselineskip
	plus .1\baselineskip
		minus .4\baselineskip
		}


% sepaqración entre párrafos.
\setcounter{tocdepth}{2}
% profundidad en la tabla de contenidos.

% \setcounter{secnumdepth}{5}
% numera todos los epígrafes hasta nivel \subparagraph en el cuerpo del documento

\usepackage{rotating}

\makeatletter
\def\@makechapterhead#1{%
  \vspace*{1\p@}%
  {\parindent \z@ \raggedright \normalfont
    \ifnum \c@secnumdepth >\m@ne
%        \huge\bfseries \@chapapp\space \thechapter
        \par\nobreak
        \vskip 20\p@
    \fi
    \interlinepenalty\@M
    \Huge \bfseries \thechapter.\space#1\par\nobreak
    \vskip 40\p@
}}
\makeatother
% Cambia el formato de los títulos de los capítulos.

\renewcommand{\mtctitle}{Contenidos}
% Cambia Contents por Contenidos en las minitocs.

%serie de tamaños de fuente: \tiny, \scriptsize, \footnotesize, \small, \normalsize, \large, \Large, \LARGE, \huge, \Huge

\addtolength{\headheight}{12pt}
\usepackage{fancyhdr}
%\pagestyle{plain} 
\renewcommand{\footrulewidth}{0pt}
\fancyhf{}

%\usepackage{glosstex}
%genera el glosario.

%\sloppy
%suaviza las rupturas de línea.

%The \flushbottom declaration makes all text pages the same height,adding extra vertical space when necessary to fill out the page. This is the standard for the book style.

% The \raggedbottom declaration makes all pages the height of the text on that page. No extra vertical space is added.

\pretolerance=1000
\tolerance=2000
%(para que no corte muchas palabras)

%\usepackage{titlesec}
%\titleformat*{\section}{\bfseries \Large}
%\titleformat*{\subsection}{\bfseries \large}


\newcommand{\um}{\ensuremath{\mu m}}

\newcommand{\subsubsubsection}[1]{\textbf{\vspace{1ex}#1\vspace{1ex}}}


%\newcommand{\figura}[3]{%
%				\begin{figure}{htp}
%								\begin{center}
%												\includegraphics[width=\textwidth]{#3}
%								\end{center}
%								\caption[#1]{#2}
%								\label{fig:eyfp}
%				\end{figure}
%}



\newcommand{\apartado}[1]{\vspace{1ex}\noindent\textbf{\emph{#1}}}
\newcommand{\titulillo}[1]{\vspace{1ex}\textbf{\Large #1}\vspace{1ex}}
\newcommand{\GC}{\textcelsius\xspace}
\newcommand{\ul}{\ $\mu$l\xspace}
\newcommand{\ug}{\ $\mu$g\xspace}
\newcommand{\infi}{$\infty$}
\newcommand{\uM}{$\mu$M}

\newcommand{\gris}[2]{\multicolumn{1}{>{\columncolor[gray]{0.9}}#1}{#2}}
%\newcommand{\gris}[2]{\multicolumn{1}{#1}{#2}}
\newcommand{\mirnam}{$\mu$iRNAM\xspace}

\newcommand{\epigrafe}[1]{\textbf{\emph{#1}}}

%\setlength{\extrarowheight}{-.5cm}


%para poner las figuras con solo \figura{nombre archivo sin extensión}{tamaño .8}{titulo corto xa el índice}{pie figura}
\usepackage{float}
\newcommand{\figura}[4]{%
   \begin{figure}[htbp]
    \begin{center}
     \includegraphics[width=#4\textwidth]{figuras/#3}
    \end{center}
		\caption[#1]{\textbf{#1: }#2}
     \label{fig:#3}
    \end{figure}}


%\usepackage[none]{hyphenat}
%		\hyphenation{mRNA mRNAs re-gu-la-do e-sen-ci-al te-ner re-fi-rién-do-nos me-ca-nis-mos di-fe-ren-cia-ción mo-le-cu-lar tes-ti-cu-lar re-gu-la-ción dis-mi-nu-ye ex-pe-ri-men-tos re-pri-mien-do ma-yor trans-crip-cio-nal se-xual Rspondin microRNAs co-re-gu-la-do-res li-gan-dos spli-cing ha-cer li-mi-ta-da a-pro-xi-ma-da-men-te miRNAs a-que-llas ma-mí-fe-ros ha-cer em-brio-na-rias des-co-no-ci-da  ma-es-tros re-tro-a-li-men-ta-ción de-sar-ro-llo su-gi-ri-ó e-jem-plo e-li-mi-na-ción me-ca-nis-mo sub-fa-mi-lia dia-na ne-ga-ti-va me-dian-te si-len-cia-mien-to exo-nu-cleo-lí-ti-ca a-ni-males rá-pi-da-men-te sen-si-bi-li-dad re-gu-la-do-res dia-nas u-san-do ad-mi-nis-tra-do te-ra-péu-ti-cas te-ra-péu-ti-co ad-mi-nis-tra-ción}

\hyphenation{mRNA miRNA miRNAs mRNAs Barrionuevo Jiménez}

%%%%%%%%%%%%%%%%%%%%%%%%%%%%%%%%%%%%%%%%%%%%%%%%%%%
%                                                 %
%                BEGIN DOCUMENT                   %
%                                                 %
%%%%%%%%%%%%%%%%%%%%%%%%%%%%%%%%%%%%%%%%%%%%%%%%%%%
\begin{document}
%\maketitle


\frontmatter

\renewcommand{\labelitemi}{$\bullet$}
\renewcommand{\labelitemii}{---}
\renewcommand{\labelitemiii}{--}


\pagestyle{empty}
\begin{center}
\LARGE
\titulo
\end{center}
\vspace{20ex}
\begin{figure}[h]
\centering
\includegraphics[width=.5\textwidth]{escudo_ugr_portada}
\end{figure}
\vspace{25ex}

\begin{center}
\large
Alicia Hurtado Madrid
\end{center}
\begin{center}
\large
Trabajo Fin de Máster

17 de Julio de 2013
\end{center}
%\newpage
\cleardoublepage
\thispagestyle{empty}
\begin{minipage}[t]{.23\textwidth}
\centering
\includegraphics[width=.8\textwidth]{logo_ib}
Instituto de Biotecnolog\'ia 
\end{minipage}
\hfill
\begin{minipage}[t]{.23\textwidth}
\centering
\includegraphics[width=.7\textwidth]{logo_ciencias}
Facultad de Ciencias
\end{minipage}
\hfill
\begin{minipage}[t]{.23\textwidth}
\centering
\includegraphics[width=.83\textwidth]{escudo_ugr_portada}
 Universidad de Granada
\end{minipage}


\begin{minipage}[b]{.23\textwidth}
\centering
\includegraphics[width=.7\textwidth]{biologia}
Estudios de Biolog\'ia
\end{minipage}
\hfill
\begin{minipage}[b]{.23\textwidth}
\centering
\includegraphics[width=.7\textwidth]{genetica}
Departamento de Gen\'etica
\end{minipage}
\hfill
\begin{minipage}[b]{.23\textwidth}
\centering
\includegraphics[width=.7\textwidth]{cibm}
Centro de Investigaci\'on Biom\'edica
\end{minipage}
\vfill
\begin{center}
\LARGE
 \titulo
\end{center}
\vfill

\begin{minipage}[t]{.35\textwidth}
Dirigida por el Doctor:

\vspace{15ex}

Dr. Francisco Barrionuevo Jiménez 

\end{minipage}
\hfill
\begin{minipage}[t]{.5\textwidth}
Memoria de Trabajo fin de Máter de la Licenciada Alicia Hurtado Madrid para la obtención del Máster en Genética y Evolución de la Universidad de Granada.

\vspace{29.7ex}

\centering 17 de Julio de 2013

\end{minipage}


\newpage
\cleardoublepage
\thispagestyle{empty}

{\rule{0pt}{85ex} El presente trabajo se ha desarrollado en los laboratorios del Departamento de Gen\'etica y del Instituto de Biotecnolog\'ia en la Facultad de Ciencias y en el Centro de Investigaci\'on Biom\'edica (CIBM) de la Universidad de Granada.
 
 
\newpage
\cleardoublepage
\input{agradecimientos.tex}
\newpage
\cleardoublepage
\null\vspace{\stretch{2}}
{
\begin{flushright}
 Dedicado a mi madre

\end{flushright}
}
\vspace{\stretch{1}}
\newpage
\cleardoublepage




\null\vspace{\stretch{2}}
{
\hfill \begin{minipage}{8cm}
\textsl{Comprender significa ser capaz de hacer.}

\begin{flushright}
Johann Wolfgang von Goethe  
\end{flushright}

\end{minipage}
}
\vspace{\stretch{1}}
\newpage
\cleardoublepage






\dominitoc
\dominilof
\dominilot
\tableofcontents
\listoffigures
\listoftables
\clearpage
%\input{abreviaturas-utf.tex}

%\input{agradecimientos.tex}
%\input{resumen.tex}
\pagestyle{fancy}


\mainmatter



\pagestyle{fancy}
\lhead{\nouppercase{\bfseries{\leftmark}}}
\rhead{\nouppercase{\bfseries{\rightmark}}}
\fancyhead[LO]{\bfseries{}}
\fancyhead[RE]{\bfseries{}}
%\fancyfoot[RO]{\thepage}
%\fancyfoot[LE]{\thepage}
\cfoot{\thepage}
\renewcommand{\headrulewidth}{0.5pt}
%estilo de las cabeceras y los pies: L=left, R=right, E=even, O=odd (impares).


%##################


\chapter{Resumen}
\input{abstract.tex}

\chapter{Introducción}
\input{determinacion-sexo_final.tex}
%\minitoc
%\input{introduccion/objetivos.tex}
%\newpage




\chapter{Objetivos}
\minitoc
%\newpage
\input{objetivos.tex}


\chapter{Material y Métodos}
\minitoc
\input{metodos_alicia.tex}
%\input{mat-met/materialYmetodos.tex}
\input{soluciones_alicia.tex}
%\newpage

\chapter{Resultados}
\minitoc
\input{resultados.tex}
%\newpage

%\input{resultados-utf.tex}

\chapter{Discusión}
\minitoc
\input{discusion.tex}
%\newpage

%\input{discusion-utf8.texs

\backmatter

\chapter{Conclusiones}
\minitoc
\input{conclusiones.tex}
%\newpage

%\chapter{Perspectivas Futuras}

%\input{conclusiones-utf8.tex}
%\input{Introduction.tex}
%\input{Aims_and_Objectives.tex}
%\input{Material_and_Methods.tex}
%\input{Results.tex}
%\input{Discussion.tex}
%\pagestyle{empty}
%\input{Conclusion.tex}
%\addcontentslineA{toc}{chapter}{Bibliography}
%\newpage
%\pagestyle{fancy}
%\bibliography{bib_manual,Bibliographydatabase}
\begin{singlespace}
	\begin{footnotesize}
		\begin{twocolumn}
\bibliography{tfm-alicia}
\end{twocolumn}
\end{footnotesize}
\end{singlespace}

%\addcontentsline{toc}{chapter}{Bibliography}
\end{document}
